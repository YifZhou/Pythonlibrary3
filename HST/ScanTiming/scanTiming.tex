\RequirePackage[l2tabu, orthodox]{nag}
\documentclass[paper=letter, fontsize=11pt]{scrartcl} % A4 paper and 11pt font size

\usepackage{fouriernc} % Use the Adobe Utopia font for the document - comment this line to return to the LaTeX default
\usepackage[english]{babel} % English language/hyphenation
\usepackage{amsmath,amsfonts,amsthm} % Math packages
\usepackage{xspace}
\usepackage{graphicx}
\usepackage{sectsty} % Allows customizing section commands
\allsectionsfont{\centering \normalfont\scshape} % Make all sections centered, the default font and small caps
\usepackage{booktabs}
\usepackage[use-xspace]{siunitx}
\usepackage{fancyhdr} % Custom headers and footers
\pagestyle{fancyplain} % Makes all pages in the document conform to the custom headers and footers
\fancyhead{} % No page header - if you want one, create it in the same way as the footers below
\fancyfoot[L]{} % Empty left footer
\fancyfoot[C]{} % Empty center footer
\fancyfoot[R]{\thepage} % Page numbering for right footer
\renewcommand{\headrulewidth}{0pt} % Remove header underlines
\renewcommand{\footrulewidth}{0pt} % Remove footer underlines
\setlength{\headheight}{13.6pt} % Customize the height of the header


\setlength\parindent{0pt} % Removes all indentation from paragraphs - comment this line for an assignment with lots of text

% ----------------------------------------------------------------------------------------
%	TITLE SECTION
% ----------------------------------------------------------------------------------------

\newcommand{\horrule}[1]{\rule{\linewidth}{#1}} % Create horizontal rule command with 1 argument of height

\title{	
  \normalfont \normalsize 
  % \textsc{university, school or department name} \\ [25pt] % Your university, school and/or department name(s)
  \horrule{0.5pt} \\[0.4cm] % Thin top horizontal rule
  \huge  WFC3 Detector Timing for Scan Mode Observations\\ % The assignment title
  \horrule{2pt} \\[0.5cm] % Thick bottom horizontal rule
}

\author{Yifan Zhou} % Your name

\date{\normalsize\today} % Today's date or a custom date

\begin{document}
\maketitle
\section{Basic Data}
Key data for calculating the read/resetting rate for WFC3 detector.
\begin{table}[!h]
  \centering
  \caption{WFC3 detector read/reset rate}
  \begin{tabular}{llll}
    \toprule
    subarray&readout time &readout time &readout speed \\
    ~&(theoretical)& (measured)&per row (\SI{}{s^-1})\\
    \midrule
    64x64&0.011&0.04&\SI{800}{} \\
    128x128&0.045&0.092&\SI{696}{}\\
    256x256&0.1825&0.257&\SI{498}{}\\
    512x512&0.730&0.832&\SI{308}{}\\
    \bottomrule
  \end{tabular}
  \label{tab:data}
\end{table}
\section{Derivation}
\newcommand{\vs}{\ensuremath{v_\mathrm{s}\xspace}}
\newcommand{\vr}{\ensuremath{v_\mathrm{r}\xspace}}
\begin{table}[h]
  \centering
  \caption{Parameter definitions}
  \begin{tabular}{ll}
    \toprule
    Names&Definition\\
    \midrule
    exptime& exposure time, the time between the start of the reset and the start of the last read\\
    $S_{0}$ & planned scan length, scan length from the start of the zeroth read and the start of the last read\\
    $L$ & array size\\
    $y_{10}$& pointing position when the reset starts \\
    $y_{1}$& pointing position when the reset past the telescope pointing position, real start of scan\\
    $y_{20}$& poiting position when last read starts\\
    $y_2$ & read end of the scan\\
    \vs & scan speed (row/s)\\
    \vr & read/reset speed\\
          \bottomrule
  \end{tabular}
  \label{tab:def}
\end{table}
\subsection{Example: down stream scan, no mid line passing}
\begin{align}
  \label{eq:1}
  & y_{10} = \frac{\vr-\vs}{\vr}y_1\\
  & y_{20} = \frac{\vr-\vs}{\vr}y_2\\
  &L_0 = y_{20} - y_{10}
\end{align}
Therefore
\begin{align}
  \label{eq:2}
  & S = y_2 - y_1 = \frac{\vr}{\vr-\vs}S_0
\end{align}
\section{List of scan length}
\begin{itemize}
\item Down stream scan, no mid line passing
  \begin{align*}
    & S = \frac{\vr}{\vr-\vs}S_0
  \end{align*}
\item Up stream scan, no mid line passing
  \begin{equation*}
    S = \frac{\vr}{\vr+\vs}S_0
  \end{equation*}
\item Mid Line passing
  \begin{equation*}
    S = \frac{\vr}{\vs+\vr}S_0 + \frac{\vs}{\vs+\vr}L - 2\frac{\vs}{\vs+\vr}(L-y_1)
  \end{equation*}
  or
    \begin{equation*}
    S = \frac{\vr}{\vs+\vr}S_0 + \frac{\vs}{\vs+\vr}L - 2\frac{\vs}{\vs+\vr}y_1
  \end{equation*}
\end{itemize}
\end{document}
%%% Local Variables:
%%% mode: latex
%%% TeX-master: t
%%% End:
